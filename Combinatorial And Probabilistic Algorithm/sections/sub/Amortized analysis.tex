\documentclass{subfiles}
\begin{document}
    Until now we have focus our attention to the single operation done by the algorithm;
    let's now suppose to have some data structure, and to operate on its data. 
    More precisely we'll considere a sequece \(S\) of operation done onto such structure.

    Just as a reference let's consider the following:
    \begin{figure}[!hb]
        \centering
        \begin{subfigure}{0.55\textwidth}
            \begin{lstlisting}[language=PSEUDO]
                procedure increaseCounter(array v, int n)
                    for i = 0 to n - 1 do
                        v[i] = not v[i]
                        if ( v[i] == 1 ) then break 
            \end{lstlisting}
        \end{subfigure}
        \caption{Algorithm for a \emph{n} bit counter.}
    \end{figure}
    Let's assume to run such algorithm \emph{n} times: one would expect that \(\BigoOf{n^{2}}\) is required in the worst case, 
    but in reality the time required is just \(\BigoOf{n}\) due to the way data may be arranged.
    \begin{definition*}
        Let \(T(n ,k)\) be the time spent in the worst case to compute the \(k\)
        operations, then 
        \[
            T_{a}(n) = \frac{T(n, k)}{k}
        \]
        represents the amortized time of the algorithm.
    \end{definition*}
    \clearpage
\end{document}
