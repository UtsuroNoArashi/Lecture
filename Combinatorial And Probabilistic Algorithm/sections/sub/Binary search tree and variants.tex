\documentclass{subfiles}
\begin{document}
    As the name suggests, a binary search tree is a search tree organized as a binary tree.
    As an example, let's consider the binary tree shown in \emph{Figure \ref{Fig:2.a}}
    whose representation as binary search tree is shown in \emph{Figure \ref{Fig:2.b}}.
    \subfile{../../figures/1 - Binary search tree.tex}
    The latter, in addition to the \emph{key} field has also three additional fields
    used to indicate the parent node, the left and right subtree roots.

    \begin{remark*}
        Given \(T\) a binary tree, we call it binary search tree if the following holds:
        for every node \(x\) in \(T\), 
        if \(y\) is a node in the left subtree of \(x\), then \(y.key \le x.key\),
        if \(y\) is a node in the right subtree of \(x\), then \(y.key \ge x.key\).
    \end{remark*}

    The above remark tells us that, if visited in order, 
    the elements of the tree appear sorted.

    \subsubsection{Red black trees}
    \subfile{../subsub/RB trees.tex}
\end{document}
