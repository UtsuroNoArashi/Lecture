\documentclass{subfiles}
\begin{document}
    \begin{figure}[!hb]
        \centering
        \begin{subfigure}{.45\textwidth}
            \centering
            \begin{tikzpicture}
            [
                tree layout,
                nodes = {draw, circle}
            ]

                \node { 6 }
                    child { node {5}
                        child {node {2}}
                        child {node {5}}
                    }
                    child {node {8}
                        child {node {7}}
                        child {node {9}}
                    };
            \end{tikzpicture}    
            \caption{}
            \label{Fig:2.a} 
        \end{subfigure}
        \begin{subfigure}{.45\textwidth}
            \centering
            \begin{tikzpicture}
                [
                    tree layout,
                    nodes = {draw, circle}
                ]

                \graph {
                    6 -> [rpIris] {
                        5 -> [rpIris] {
                            2,
                            5' [as=5]
                        },
                        8 -> [rpIris] { 
                            7,
                            9
                        }
                    },

                    2 -> [bend left, rpFoam] 5,
                    5' -> [bend right, rpFoam] 5,
                    7 -> [bend left, rpFoam] 8,
                    9 -> [bend right, rpFoam] 8,
                    5 -> [bend left, rpFoam] 6,
                    8 -> [bend right, rpFoam] 6
                };

            \end{tikzpicture} 
            \caption{}
            \label{Fig:2.b} 
        \end{subfigure}
        \caption{A simple binary tree, alongside its binary tree representation.
        Here with \textcolor{rpIris}{\rightarrow} we indicate the pointers 
        to the left and right child, 
        while with the \textcolor{rpFoam}{\rightarrow} we indicate the pointer to the parent.}
    \end{figure}
\end{document}
